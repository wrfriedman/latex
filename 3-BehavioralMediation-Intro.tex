\documentclass[11pt]{amsart}
\usepackage{geometry} 
\geometry{letterpaper, margin=1in}
\renewcommand*\rmdefault{ptm} % set to times

\usepackage{natbib}
\usepackage{mathtools}
\usepackage{graphicx}
\usepackage{caption}
\usepackage{subcaption}
\usepackage{wrapfig}
\usepackage{sidecap}
\usepackage{textcomp}
\usepackage{lscape}
\usepackage{framed}
\usepackage{enumitem}
\usepackage[ampersand]{easylist}

\usepackage[final]{pdfpages}

\usepackage [english]{babel}
\usepackage [autostyle, english = american]{csquotes}
\MakeOuterQuote{"}

\graphicspath{
{/Users/Whitney/Dropbox/Aerial/Dissertation/images/}}
\title{Chapter 3: \\ Interactions that Mediate Alliance Relationships \\ General Introduction and Methods}
\date{}
%%% BEGIN DOCUMENT
\begin{document}

\maketitle

%%%%%%%%%%%%%%%%%%%%%
\section{Introduction}
%%%%%%%%%%%%%%%%%%%%%

In this section I aim to address the question \emph{how are alliance relationships mediated through social interaction?} Behavioral observation of alliances \emph{in situ} suggests that coordinated social behavior is a fundamental aspect of mate competition and consortship, two principle activities that male allies engage in. During intense competitions between second-order alliances, males can be observed coordinating in both space and time as they engage in behaviors such as charging to attack a single individual. During consortships, allies can be observed coordinating in space and time when they follow behind a female dolphin, produce intricate sequences of mirrored synchronous behavior around a female consort, or simply to surface or dive in synchrony. Of such behaviors, only synchrony, which has a reliable surface component, has been quantified due to the inherent challenge of reliably quantifying the many other kinds of behavior that often occur underwater and out of the sight of boat-based observers. In this study we use aerial videography paired with boat-based data collection to surmount this challenge \ref{fig:aerialView}. Parsimony suggests that males should enact (communicate) their relationships via the same mechanisms critical to their success as allies. 

The fundamental hypothesis examined by this chapter is that males use \emph{coordinated social behavior} to mediate their relationships. I look at two contexts to understand the variation in production of coordinated social behavior: first-order male allies with a female consort, and third-order fusion events occurring between two first-order male consortship groups from different second-order alliances. I test the prediction that males who are most successful (have the highest consortship rates) also engage in the highest rates of social coordination. I also examine the relationship between consortship rate and association rate, and test whether males who have the highest association rates also engage in the highest rates of social coordination. Finally, I test the prediction that fusion events are accompanied by higher rates of coordinated social behavior and social vocalizations than after 10-15 minutes of association among third-order allies. The list of hypotheses below build on each other consecutively to address these predictions.

%%With all hypotheses, the key thing is how they fit in a THEORY.  So, as you test these hypotheses, you should be specifying how they constitute a theory of dolphin social organization. 
%\noindent The network analysis describes an observed set of relationships defined by spatial association. Here I examine how such relationships are \textit{mediated} through social interaction. I predict that dyads will show differences in the type and context of their interactions, and that these differences will relate to differences in association strength and reproductive success. That is, I predict that its not just being together that makes an alliance: it is \textit{what}, \textit{when}, and even \textit{how} allies interact that describe these relationships. 
%%
%\footnote{Among primates, extensive research has revealed the importance of social interactions for understanding the nature of social relationships. Interactions can have immediate consequences, as in unrelated female vervet monkeys for whom grooming prior to a call for help increased affiliates response duration \citep{seyfarth+cheney:1984}, or among male chimpanzees who directed grooming towards potential allies in moments of intense social conflict \citep{nishida:1983}, but see \citep{deWaal:1984}. In other contexts, interactions may be used to test or strengthen bonds between individuals \citep{zahavi:1977}. For example, eye-poking between capuchins [20] or sexual greetings among adult male baboons may be a test of the strength of the relationship between individuals. In the latter study, the strongest male alliance was also the only dyad that showed complete symmetry in their exchange of sexual greetings \citep{smuts:1990}. Finally, certain kinds of interactions among individuals can influence the eventual reproductive success of the actors. For example, in adult female savannah baboons, time spent grooming and associating with other adult females is correlated with reproductive success \citep{silk:2003}.}
%%
%Beyond this, I predict that the \textit{kinds} of interactions that strong allies use to mediate their relationships should bear parsimonious relationship with the kinds of interactions allies require to succeed in the fundamental context of their existence, namely mate competition and consortships. Males should \emph{enact} \citep[c.f.][]{VRT:1991} their relationships via the same types of interactions required for reproductive success. This prediction rests on the well-supported theory that alliances are primarily about reproductive success: the existence of alliances means that males are in direct reproductive competition with the same individuals they require for success in mate competitions. Alliances are thus costly, and should not exist at all unless the benefits of alliance formation outweigh their costs. If alliances are primarily about reproductive success, then the types of interactions they use to mediate their relationships may also communicate something about their probability of increased success, both to the individuals involved in such interactions, as well as in advertisement to others. Such displays occur in various forms among social mammals: a simple example is the pilo-erection seen among male primates pre-empting or during fights \citep[e.g.][]{perry:1998}, or the low-level dominance poses used by canines to communicate the nature of their relationship and pre-empt fights \citep{bekoff?}. In both examples the observed behavior is a \textit{synecdoche}: a physical part of a more intense behavior used by individuals when the nature of the interaction is more intense. 
%
%The hypotheses below build consecutively on each other. H1-H2b establish the variation in the production of coordinated social behavior among male allies, and relate that variation to the strength of the alliances as measured by their association consortship rate. H3 and H4 examine the context of production, providing a more direct test to the prediction that males use social coordination to mediate their alliance relationships. Finally, H5 examines the directionality of petting events to test for the presence of asymmetric relationships.


\begin{easylist}
\ListProperties(Hide2=10,Progressive*=.5cm,Start1=1,Space=2mm)
\vspace{5mm}
& I look at the context of consortship groups (1st-order males with a female) to understand how variable the production of social coordination is among male allies, and relate this to the rate at which each dyad associates and consorts together. 
&& H1: There are differences in the production of coordinated behavior among male allies. 
&& H2a: Males with higher dyadic \textit{association} indices (those who associate together most frequently) also engage in coordinated social behavior more frequently than males with lower dyadic association indices
&& H2b: Males with higher dyadic \textit{consortship} indices (those who consort together most frequently) also engage in coordinated social behavior more frequently than males with lower dyadic consortship indices

& Then I look at the context of fusion events between two consortship groups from different 2nd order alliances (3rd-order fusion events) to further understand how allies use social coordination to mediate their relationships. 
&& H3: The production of SC among third-order allies will be higher at fusion events than 10+ min post-fusion. 
&& H4: Third-order fusions will be accompanied by higher rates of acoustic vocalizations (whistling) than 10+ min post-fusion

& I look at the directionality of petting events scored in both contexts to address the hypothesis that some individuals initiate more petting bouts than others, and that those that initiate petting bouts consort less frequently than those individuals who they direct petting towards.
&& H5: When the directionality of a petting event can be discerned, males direct petting towards partners who have a higher consortship rate than themselves. 

& To explicitly describe the \textit{process} of coordination, I use a limited set of the scored behaviors (displays or synchrony), to conduct a fine-scale descriptive analysis of the coordination of visible sensorimotor resources through space and time. 
\vspace{2mm}
& Finally, I draw on the the analyses conducted this chapter to describe the \emph{cognitive complexity} exhibited by the male alliance network in terms of the range of interactions observed, the context and embeddedness of their production, the process of interactions, and how such processes relate to the structure and dynamics of the web of relationships exhibited by the alliance network.
\end{easylist}

%\subsection{Background}
%The study of social interactions in non-humans?

%%%%%%%%%%%%%%%%%%%%%
\section{Data Collection}
%%%%%%%%%%%%%%%%%%%%%

\subsection{Site}
Data for this study were collected as part of a longitudinal study of bottlenose dolphins (\textit{Tursiops} sp.) in Shark Bay, Western Australia. Research at this site has been conducted since the early 1980's, and individuals are well-habituated to boat-based researchers. Data analyzed for this study were collected during the peak mating season in the the austral spring (September - December) in 2013 and 2014. 

\subsection{Subjects}
The subjects of this study were 20 males found within three "second-order" alliances: KS, PD, and RR (see Chapter 2).  All three groups were highly accessible as they frequented the waters nearest the field station in Monkey Mia (Figure \ref{fig:ranges}). They are the most intensively observed alliances so there is an extensive database on associations and the composition of 1st order-alliances within these groups. Males in all groups form trios to consort females, almost exclusively. The original study design contained only the 15 KS and PD males due to their long-term history of association and known third-order alliance relationship, however during the course of the study the RR males began associating with these two groups and are treated as probable third-order allies even though we do not yet have a "smoking gun" observation of the RR males joining sides with either KS or PD during a large-scale mate competition. 
% Despite this, evidence strongly suggests that RR males are third-order associates of both KS and PD. [Review the evidence for RR third order].

\subsection{Focal Follows}
\subsubsection{Data Collection}
One-hour boat-based focal follows were conducted on each of the fifteen males in the KS and PD alliances. Focal males were selected using a daily randomized list of individuals from each alliance. Since we were at the mercy of encounter to find the individuals, when we did encounter a KS or PD subgroup, we followed the first individual from the list who was present, and who we had followed least frequently. In effort to insure independent samples, in the case that we finished a focal follow with another KS or PD subgroup in sight, we switched to a new KS or PD subgroup, and selected the next focal in the same manner as described above. If no new subgroup was in sight, we waited twenty minutes before beginning the next focal follow on the next individual in the randomized list from within our group.

During each one-hour focal follow, behavioral and ecological data were recorded simultaneously using a range of sensors: 
\begin{enumerate}[label*=\arabic*.]

\item \emph{Aerial video} was collected using a tethered helium aerostat. The 2.3m diameter Allsopp "Helikite" was tied to the boat
% METHODS FROM POSTER HERE !! 
Tethered helium aerostats have been used to record foraging behavior of bottlenose dolphins \emph{Tursiops truncatus} off the Florida coast \citep{nowacek:2002}, activity budgets of free-ranging dugongs \emph{Dugong dugong} %ck this?
and their responses to anthropogenic noise \citep{hodgson:2004}, and leadership among bottlenose dolphins in the Lower Florida Keys \citep{lewis:2011}. 

\item \emph{Boat-based video} was collected using a hand-held Canon Vixia HF11 camcorder. 

\item \emph{Acoustic recordings} were collected using a towed hydrophone fabricated at the SIO lab % named... 
equipped with one low-frequency transducer (P190) and one high-frequency transducer (HS150). The frequency response curve for this hydrophone is shown in Figure ~/ref{fig:hydrophoneResponse}. Acoustic data were recorded using a Fostex FR-2 Memory Recorder % more details

\item \emph{Vocal annotations} were used to continuously record the identities of individuals in order to later identify them on video records. 

\item \emph{5-minute behavioral and ecological interval samples} were collected by trained boat-based observers. Figure \ref{fig:ethogram} shows a completed ethogram. The columns, labelled 1-8, correspond to the following kinds of data: 
\begin{enumerate}[label*=\arabic*.]
\item (a) Time of 5-minute scan. (b) Number of dolphins within the group (using 10-meter chain rule; \citep{smolker:1992} (c) The nearest male neighbor (NN) of the focal individual. (d) The category of notes being recorded (e.g. '5' means 5-minute scan, 'FIZ' refers to a departure, 'FUZ' refers to an arrival), or frame number of photo with notes. 
\item The activity pattern of the focal and his nearest neighbor, including the distance ('dist') between them.
\item Dive synchrony - whether the focal and NN made their last dive of the bout synchronously ('Y'), not synchronously ('N') or almost synchronously ('al') \citep[see][]{connor:2006}.
\item The relative orientation of focal and NN at the time of the beep. 
\item The focal's position within the group (L=leading third, M=middle third, T=trailing third). 
\item General characteristics of group activity (Gr Act) including: traveling, resting, socializing, foraging, or unknown. Movement (Mvmt) = straight (whole group maintains same heading), meander (group moves in multiple directions, but together), mill (individuals within the group move in mixed directions). Heading (Hdg) = cardinal direction of the group. Speed (Spd) = very slow (\textless1mph), slow (1-2mph), cruise (2-3mph), moderate (3-4mph), fast (4-6mph), and blasting (\textgreater6mph).  
\item Ecological data: depth, bottom substrate (sandy or seagrass 1-4), habitat (1=beach, 2=offshore shallows, 3=embayment plain, 4=channel, 5=edge of channel, 6=flats), beaufort. 
\item Group composition/Notes: qualitative description of activity and social coordination events.
\end{enumerate}

\item \emph{Positional} data were continuously recorded using a Magellan eXplorist 610 GPS unit.
\item \emph{Depth} data were recorded at 5 or 15 minute intervals on the ethogram. The depth sounder was turned off between intervals to reduce the amount of additional noise captured on the hydrophone.
\end{enumerate}

\noindent All signals were aligned at the start of each focal session using a film clapperboard. Data were synchronized for review using ChronoViz \citep{fouse:2011}, a software tool for visualization and analysis of multi-modal data (Figure ~\ref{chronoviz}). 

\subsection{Consortship Data}
\subsubsection{Data Collection}
Consortships are recorded using an all-occurrence protocol \citep{altmann:1974}. These events take precedence over other sampling methods as they are of critical import for defining the relationships among allies. A 'consortship' is defined by any single occurrence or combination of the following events \citep{connor:1996a}: 

\begin{enumerate}
\item A female is captured by an alliance of males
\item A female attempts to escape by rapidly swimming ('bolting') from an alliance of males
\item At least one of the males produces a vocal threat called a 'pop' that induces the female to remain close. These are often produced when the male has his head at the surface of the water and the pops can be heard in air by observers \citep{connor:1996b}
\item At least one of the males directs physical threats or aggression toward the female
\item Teams of two alliances attempting to take the female from the focal alliance \citep{connor:1992a, connor:1992b}
\item A clear pair or trio of males maintain association with a female for at least one hour, or across multiple surveys (total time longer than 1hr) % citation
\end{enumerate}

\subsubsection{Data Analysis}

\subsection{Behavioral Sampling of Video Data}
\subsubsection{Data Collection}
% Interval sampling 
A group instantaneous-sampling method \citep{altmann:1974} was developed and employed to capture rates of interactions among male dyads. The interval used was 5-seconds: this interval is longer than the length of the shortest event sampled and therefore allows for the capture of changing behavior over time, but is short enough to consistently track the movement of individuals relative to one-another. %\citep{altmann:1974}, mann:1999 Notes: group sampling, intervals sufficiently short, benefits of intervals over continous sampling. 
In addition to the benefits of interval over all-occurrence and one-zero sampling, this method is particularly well-suited to the aerial video dataset. At each sampling point, all dolphins who are present for observation (see below) are recorded, and only the behaviors among those visible individuals are scored. %In a sampling context in which there are always unknowns...suitable..

% Contexts:
Three social contexts were scored from the aerial video dataset using this method: (1) Male trios (first-order alliances) with female consorts ("consortships"), (2) The fusion of two first-order alliances, with female consorts, from two different second-order alliances ("third-order fusions"), and (3) 10+ minutes after third-order fusions ("post-fusions").

% Present for observation:
At each five-second interval, the identity of all individuals who were "present for observation" were recorded. To be present for observation, an individual had to be distinguishable on the aerial video data as an "identifiable dolphin-like shape", including resolution of flukes, rostrum, and/or pectoral fins, within two frames (0.067s) of the interval frame. Limiting factors include the vertical visibility through the water, moving surface sun glare, and the framing of the group on the aerial video. 

% Ethogram & Data sheet
Behaviors were scored within five categories (see ethogram in Figure \ref{fig:ethogram} for further detail on categories, behaviors scored, mutual exclusivity rules, and differentiating actors and recipients):
\begin{enumerate}
\item{\emph{Synchrony}} was scored if 2+ dolphins broke the surface / dove within 9 frames (0.033 seconds) of each other. Frames were counted between the first emergence / total disappearance of dorsal fins, or heads if dorsal fins were not visible or of the same size / shape \citep{connor:2006}. This behavior was primarily scored using the deck video, but could be scored by counting frames between the first evidence of exhalation (spray) if visible only on the aerial video. Finally, the defining part of the event (breaking surface / disappearing) must have occurred within 1-second of the interval in order to be recorded. 
\item{\emph{Polyadic Displays}} were scored if 2+ dolphins engaged in a spatially and temporally coordinated sequence of movements. The simplest display is a "tango" where two individuals in parallel engage in two or more quick turns in unison while maintaining their parallel orientation. More complicated displays include mirrored activity, and mirrored or parallel activity around a third individual. Displays often included, but were not limited to synchronous surfacing or diving. Synchrony (as described above) and displays are mutually exclusive. Records indicate which individuals coordinated their movements with each-other (actors), and whether they were coordinating their movements around a third individual (recipient), as well as whether the display included contact.
\item{\emph{Non-agonistic Contact}} events were scored if 2+ dolphins contacted via petting (using pectoral fins) or rubbing of other appendages. Events in which dolphins were overlapping spatially but petting or rubbing could not be confirmed were scored as "distance-zero" (D0). Behaviors were distinguished by the body parts in contact, including ventral-ventral rubbing or mounting, either of which may indicate copulation. Records indicate actor and recipient when distinguishable.
\item{\emph{Formations}} are salient patterns of spatial orientation that are often held for periods of time. Four types of formations were scored: individuals positioned in parallel ($<5^{\circ}$), less than 1/2 of a body-length offset and within one body-length's distance from their neighbor ("abreast"), individuals positioned in parallel ($<5^{\circ}$), less than 1 body-length offset and greater than two body-length's distance from their neighbor ("lateral line"), individuals positioned in parallel but around another individual ("formation"), and individuals returning from a distance to a position around a single individual ("converging"). Records indicate actors and recipient if applicable.
\item{\emph{Agonistic}} interactions are forceful, directed events that produce a salient reaction from the recipient. Reactions including rapidly swimming away, fluking up, or lying passively at the surface. Records indicate actors and recipients, as well as whether the event was dyadic or polyadic. 
\end{enumerate}

All three contexts were scored using the same ethogram and methods described above. Details specific to each context are summarized in table \ref{table:contexts}.  

% IORS FOR BEHAVIOR!! 
IORs for identifying behavior in the above categories: 
Percent Agreement: 0.82
Cohen's Kappa: 0.73


\begin{table}
\begin{tabular}{| p{2cm} | p{5cm} | p{5cm} | p{2.6cm}|}
\hline
Context & Subjects & Sampling Began & Sampling Period 
\\ \hline
\emph{Consortships}
& Groups of 2-3 males (first-order allies) and confirmed female consort, all other dolphins at least 10-meters away. 
& The first five-minute interval after the start of the focal follow in which 2+ group members were present for observation.
& 10 min
\\ \hline
\emph{Third-Order Fusions}
& Two first-order consortship groups (2-3 males plus confirmed female consort) from different second-order alliances.
& First evidence of fusion, either (A) the first frame in which the two subgroups were within 10 meters, and at least 1 member from each group was present for observation, or (B) if fusion is not captured on aerial, the exact time observers recorded the fusion event.
& 2 min
\\ \hline
\emph{Post-Fusions}
& Two first-order consortship groups (2-3 males plus confirmed female consort) from different second-order alliances.
& The first frame $>$10 minutes after a thrid-order fusion in which 2+ group members were present for observation. Samples are paired with preceding third-order fusions when data are available.
& 2 min
\\\hline
\end{tabular}
\caption{}
\label{table:contexts}
\end{table}

%
%listed below: \\
%\begin{easylist}[itemize]
%& \emph{Consortships}
%&& Subjects: Groups of 2-3 males (first-order allies) and confirmed female consort, all other dolphins at least 10-meters away. 
%&& Sampling began: The first five-minute interval after the start of the focal follow in which 2+ group members were present for observation.
%&& Sampling period: 10 minutes
%
%& \emph{Third-Order Fusions}
%&& Subjects: Two first-order consortship groups (2-3 males plus confirmed female consort) from different second-order alliances.
%&& Sampling began: The first frame in which the two subgroups were within 10-meters, and at least 2 members from each group were present for observation.
%&& Sampling period: 2 minutes
%
%& \emph{Post-Fusions}
%&& Subjects: Two first-order consortship groups (2-3 males plus confirmed female consort) from different second-order alliances.
%&& Sampling began: The first frame $>$10 minutes after a thrid-order fusion in which 2+ group members were present for observation. Samples are paired with preceding third-order fusions when data are available.
%&& Sampling period: 2 minutes
%\end{easylist}

\subsection{Acoustic Sampling}
Acoustic data were sampled over the same periods as third-order fusions and post-fusions. A one-second interval sampling method was employed to capture the rate of acoustic production during these contexts. Acoustic data were reviewed using spectrographic displays (Fast Fourier Transform size: 1024, weighting function: Hanning window, time resolution: 3s, sampling rate: 192 kHz) using the free, open-source software Audacity. On each one-second interval, the observer recorded the presence of whistles ("W"), burst-pulse calls ("B"), pops ("P"), or echolocation ("E"). 

%Those recording segments in which engine noise exceeded 2 kHz were discarded from the analysis. King & Janik 2014

% Whistles have a contour and are continuous through time. 
% Burst-pulse calls are punctuated - if a series was on-going at the 1-second interval, was called "present" regardless of whether that exact point in spectral space indicated signal...
% Produce / see figures... 
% IORS

%%%%%%%%%%%%%%%%%%%%%

% Summary of focal hours / male? 

%%%%%%%%%%%%%%%%%%%%%

%%%%%%%%%%%%%%%%%%%%%
%\section{Discussion}
%%%%%%%%%%%%%%%%%%%%%

%%%%%%%%%%%%%%%%%%%%%
%\section{Figures}
%%%%%%%%%%%%%%%%%%%%%

\begin{figure}
  \centering
    \includegraphics[width=1\textwidth]{AerialBenefits}
  \caption{Aerial and Side-angle Cameras} % elaborate
  \label{fig:aerialView}
\end{figure}

\begin{figure}
  \centering
    \includegraphics[width=.6\textwidth]{behaviorCategories}
  \caption{Behavioral Categories} % elaborate
  \label{fig:behaviors}
\end{figure}

\begin{figure}
  \centering
    \includegraphics[width=1\textwidth]{ranges}
  \caption{Ranges, fixed kernal density, reproduced from \citet{randic:2012}} % elaborate
  \label{fig:ranges}
\end{figure}

\begin{figure}
  \centering
    \includegraphics[width=1\textwidth]{focal}
  \caption{Behavioral ethogram and focal scan data collected in 2012. See in-text description.} % elaborate
  \label{fig:ethogram}
\end{figure}

\begin{figure}
  \centering
    \includegraphics[width=1\textwidth]{710_130503_Calibration_Plot}
  \caption{Hydrophone response curve} % elaborate
  \label{fig:hydrophoneResponse}
\end{figure}

\begin{figure}
  \centering
    \includegraphics[width=1\textwidth]{Ethogram}
  \caption{Behavioral Ethogram} % elaborate
  \label{fig:ethogram}
\end{figure}

%%%%%%%%%%%%%%%%%%%%%
% REFERENCES % 
%%%%%%%%%%%%%%%%%%%%%
\clearpage
\bibliographystyle{apalike}
\bibliography{/Users/Whitney/Dropbox/Articles/bibliography-wf}

%%%%%%%%%%%%%%%%%%%%%
\end{document}