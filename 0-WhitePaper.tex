\documentclass[11pt]{amsart}
\usepackage{geometry} 
\geometry{letterpaper, margin=1in}
\renewcommand*\rmdefault{ptm} % set to times

\usepackage{natbib}
\usepackage{mathtools}
\usepackage{graphicx}
\usepackage{caption}
\usepackage{subcaption}
\usepackage{wrapfig}
\usepackage{sidecap}
\usepackage{textcomp}
\usepackage{lscape}
\usepackage{framed}
\usepackage{enumitem}
\usepackage[ampersand]{easylist}

\usepackage [english]{babel}


\graphicspath{
{/Users/Whitney/Dropbox/Aerial/Dissertation/images/}
{/Users/Whitney/Dropbox/Aerial/DataAnalysis/Network/KsPdRr09-14/FinalAnalysis/Figures/}
{/Users/Whitney/Dropbox/Aerial/DataAnalysis/Network/KsPdRr09-14/FinalAnalysis/Networks/}
{/Users/Whitney/Dropbox/Aerial/DataAnalysis/Network/KsPdRr09-14/FinalAnalysis/Stats/0914_Stats/}
{/Users/Whitney/Dropbox/Aerial/DataAnalysis/Network/KsPdRr09-14/FinalAnalysis/Stats/2y_Stats/}}

\title[Social and cognitive complexity among alliances of male bottlenose dolphins (\emph{Tursiops sp.})]{Social and cognitive complexity among alliances of male bottlenose dolphins (\emph{Tursiops sp.}): A Brief Introduction}
\date{}
%%% BEGIN DOCUMENT
\begin{document}

\maketitle

%%%%%%%%%%%%%%%%%%%%%
\section{Background}
%%%%%%%%%%%%%%%%%%%%%

Male bottlenose dolphins in Shark Bay, Western Australia participate in up to three levels or \textit{orders} of alliance formation within a large open society of resident dolphins. \textit{First-order} alliances are comprised of pairs and trios of adult males who work together to sequester and consort estrus females \citep{connor:1992a}. Most adult males also belong to teams of 4-14 individuals, or \textit{second-order} alliances, who work together to take females from other males and to defend against such attacks \citep{connor:1992a, connor:1992b}. In 2011, a \textit{third-order} of alliance formation was described in which two second-order alliances coordinated against other males during competitions over females \citep{connor:2011}. 

Such nested structure where interactions among the same components at different scales result in functional differences meets the same formula for \textit{complexity} used to describe the neural networks that constitute the human brain \citep{tononi:1994}. The nested structure of alliances in Shark Bay may be the most complex outside of humans, and while much work has been done to describe the function of first- and second-order alliances, to date little is known about third-order alliances, how the structure of alliance networks are maintained, or how they respond to change.

In R.A. Hinde's classic scheme, a population's social structure is described by its pattern of social relationships, which, in turn, are comprised of a history of interactions among individuals \citep{hinde:1976}. In this dissertation I traverse Hinde's levels to examine the relationship between social complexity through quantitative analysis of a third-order alliance network, and social cognition through fine scale observations of behavioral interactions among allies. I aim to make two kinds of contributions (A) to lay a few boards in the growing methodological bridge towards better understanding the relationship between, and co-evolution of, complex societies and complex cognition, and (B) to further our understanding of dolphin society and cognition. There are two principle questions that guide the research in this dissertation: (1) how are third-order alliances structured, what are their dynamics, and how do they respond to change? (2) How do the particular kinds of behavioral interactions that allies engage in relate to the observed structure and dynamics of the alliance network?

\begin{figure}[h]
  \centering
    \includegraphics[width=.8\textwidth]{thesis-in-a-nutshell}
  \caption{The theory, methods, and big picture questions discussed in this dissertation}
  \label{fig:dissertationNutshell}
\end{figure}

\subsection{Social and Cognitive Complexity}

In the 1960's and 70's, advocates of the \textit{Social Intelligence Hypothesis} proposed that intelligence, rather than % In addition to?
evolving in the context of increased foraging or technical skills, was linked in evolutionary ratchet with increasingly complex societies \citep{jolly:1966,humphrey:1976}. Jolly wrote, ''since [social primates'] dependence on the troop both demands social learning and makes it possible, social integration and intelligence probably evolved together, reinforcing each other in an ever-increasing spiral" \citep[p 504]{jolly:1966}. 

In the late 1980's, the Social Intelligence hypothesis morphed into the Machiavellian Intelligence hypothesis, so called because individual success within complex societies appeared to be "...most effectively promoted by seeming altruistic, honest, and prosocial" \citep{byrne:1988, byrne:1997b}. While research conducted under this hypothesis has found a close relationship between social skills, group complexity, and brain size \citep{byrne:1997a, byrne:1997b}, the excitement garnered in the 80's and 90's over this approach has since dwindled. Key questions focused on what mental representations individuals formed about others' relationships and perspectives, including whether they participated in deliberate deception of conspecifics, and whether they attributed and acted upon surmised belief states of others \citep{byrne:1988, premack:1978, forster:2012}. The focus on using behavior as a "window into" the internal representational states of subjects was echoed throughout first generation cognitive science, but has since been abandoned by many researchers in favor of a more biological account of cognition. 

Over the last thirty years, the field of cognitive science has seen a revolution of methodological approaches to the study of cognition. As more is understood about the continuity of brains, bodies, and worlds dynamically coupled in cognitive processes, researchers have been challenged to examine their units of cognitive analyses, question the absolute centrality of the individual mind in cognitive systems, and increasingly move towards an understanding of cognition as biological processes continuously produced by richly interdependent ecological webs. % Citations

A fundamental insight from this revolution is that social interactions are not simply \textit{residua} of brain-bound cognitive events; but are themselves cognitive processes constituted by richly interconnected ecosystems spanning brains, sensorimotor systems, bodies, and social interlocutors \citep{hutchins:1995b,hutchins:2006}. On this view, social interactions are at least as descriptive of such distributed cognitive processes as the recordings we might take of an individual's neural activity. The careful description of such interactions constitute data points on cognitive events that can be taken on their own right, and may also be used to carefully generate hypotheses about the kinds of sensorimotor and perceptual capabilities a set of organisms must possess of in order to produce the observed social interactions \citep{byrne:2001, johnson:2001, johnson:2001b, king:2003, forster:2006, forster:2012}. 

Re-visiting Hinde's levels with this insight in mind we find renewed methodological traction for the study of the relationship between social and cognitive complexity. Rather than the invisible box left off Hinde's handy diagram, we find that cognitive events are baked right in: they are the interactions. In this dissertation, my aim is to explore and explicate the relationship between social complexity, taken as a quantifiable property that describes a particular kind of patterning of relationships of in an animal society, and cognitive complexity, taken as the context, process, and repertoire of social interaction. 

% ADD: semiotic resources, forster

\section{Social Complexity \& Network Analysis}
% How does this theory of stability relate to social complexity?
% Theories: organization (social roles), and dynamics (stability)
%In Shark Bay, participation in alliances is the predominant strategy governing male reproductive success \citep{krutzen:2004}. Such participation is not generally transitory, but often reflects a long history of investment: relationships among male dolphins may begin to form as early as the calf period \citep{stanton:2011}
%but it appears to take years beyond sexual maturity (\texttildelow12y) for alliances to stabilize in terms of group membership, and to become successful at group competitions. Thus, by the time males are in their reproductive prime (\texttildelow18-35y), % data
%most have already spent years investing in their alliance relationships. Given the time it takes for alliances to form, and the critical nature of such relationships for males reproductive success, it makes sense for alliances to purse a strategy of stability. Indeed, my colleagues have shown that once males assimilate within a second-order alliance, such groups can be stable for over thirty years \citep{connor:2015}. Stability in first-order alliance composition has been shown to correlate with higher consortship rates \citep{connor:2001}, and may also reflect a local solution to the composition and structure of relationships that maximizes a second-order alliance's ability to succeed during mate-competitions. However, as with any natural system there are likely to be sources of instability: males may shift alliances in pursuit of higher reproductive success, mortality can cause local or widespread shifts in relationships, and environmental perturbations can change the availability of forage and produce secondary effects on alliance relationships. 

The nested structure of alliances in Shark Bay may be the most complex outside of humans, and while much work has been done to describe the function of first- and second-order alliances, to date little is known about third-order alliances, how the structure of alliance networks are maintained, or how they respond to change. In this chapter I describe the structure, function, and dynamics, and complexity of a third-order alliance network over a six-year period of time. During the study period, four of the original twenty-four males in the study group 'disappeared', and are presumed to have died, and new third-order relationships were forged. Coincident with the observed shifts in social relationships, the larger Shark Bay ecosystem experienced a cascade of changes including widespread seagrass die-off associated with an one of the strongest \emph{La Ni\~{n}a} events ever recorded in the region \citep{pearce:2011,thomson:2014}.  

Three theories about dolphin social organization guide the work in this chapter. The first is that individuals engage in different kinds of social roles: I expect to find that some individuals are more gregarious than others, and that such social propensity is related to reproductive success. Beyond this, I predict that there are some individuals who play a greater role in the maintenance of third-order relationships, while others may play a greater role in the maintenance of second-order relationships. These predictions are grounded in life history theory, evidence for roles among other social mammals, and the Darwinian concept of phenotypic variation.
% Background: social roles, life history theory, gregariousness / social differentiation / phenotypic variation
The second theory is that second-order alliances pursue a strategy of stability and maintained/maximized levels of reproductive success. Thus, they should respond to change by re-organizing in ways that maintain their alliance relationships and reproductive success. The organizational structure of a second-order alliance should relate to its resilience to change: groups with more individuals and more heterogeneity of relationships %(and therefore higher entropy) 
have pre-existing architecture for re-organization, whereas groups with fewer individuals and homogeneity of relationships are option-limited when group members are lost. Differences in social organization may represent a strategic trade-off: groups with more stable first-order alliances may experience higher reproductive success when in that stable solution, but risk dramatically reduced reproductive success due to lack of resilience; groups with more heterogeneity of relationships may have median levels of reproductive success but maintain those levels through change. This theory is supported by evidence for long-term participation in second-order alliances, and a previously reported relationship between first-order stability and consortship rate (a proxy for reproductive success) \citep{connor:2001}.
% Background: long-term investment, group membership. Stronger 1st order allies have higher CSR.
The third theory is that structural complexity relates to cognitive complexity. In this chapter I examine the degree to which the patterning of relationships exhibited by the alliance association network meets well-cited criteria for structural complexity. The following chapter will examine how patterns of interactions among allies relates to the structure described here. % Risks circularity. Keep trying to better explain this. 

In this analysis, I start by describing the global structure of a third-order alliance network using measures from social network analysis. I evaluate the centrality of individuals based on their network position and address whether there are particular individuals who play a greater role in the mediation of second- and third-order alliance relationships. I then subset the six-year dataset into two-year intervals to describe the shifts in the structure of alliance relationship in response to the "disappearance" (probable death) of four males in the third-order alliance network, and the warm water anomaly described above. I examine the prediction that alliances respond to change by re-organizing to find new stable solutions, and that such re-organization includes filling vacant "social roles" critical for maintaining alliance relationships. I test the hypothesis that second-order alliances who show greater entropy are more resilient to change. Finally, drawing on the analyses in this chapter, I evaluate the complexity of the alliance network using a quantitative measure (developed in the neurosciences) that may be useful in studies of comparative social network complexity. 

\vspace{6mm}
\begin{easylist}
\ListProperties(Progressive*=.5cm,Space=2mm)
% Structure - Descriptive
& What is the structure of second- and third-order alliance?
&& I describe the global structure of the second- and third- order alliance using hierarchical clustering analysis, network analysis metrics, Shannon Entropy, and Complexity ($C_N$)
&& I describe the differences in second-order alliances in terms of their entropy. Some second-order alliances show a more 'crystalline' or modular distribution of relationships, while others show more variation in dyadic relationship and should therefore show higher entropy.
&& I examine the question of whether there are individuals who exhibit different social roles in the mediation of second- and third-order alliances by describing individuals in terms of their network centrality. I introduce a few variations on the classic centrality measures that allow for a more direct assessment of this question.
% Dynamics - Descriptive
& What are the dynamics of second- and third-order alliances?
&& I describe the change in social networks from one time period to the next using the same metrics for describing global structure and roles indicated in (1). 
&& I use a permutation analysis to test the hypothesis that the alliance network significantly changed from one two-year time period to the next. 
&& I treat the loss of individuals as a natural perturbation study and describe the observed versus expected changes in the alliance network.
&& I use the quantitative metrics above in combination with observational data to provide a detailed descriptive analysis of the observed change in alliance relationships.
% Predictive
& What is the function of third-order alliances?
&& I use the prior analyses to examine the prediction that that males who loose first-order allies respond by forming new first-order relationships within their alliance network, selecting from third-order allies if no second-order allies are available. 
& What features of alliance relationships best predict the observed patterns? 
&& Life history theory predicts that prime aged males should be investing maximal resources into increased reproductive success. In this population, the predominant strategy for increasing reproductive success is through alliance formation, so all males should be putting equal effort into forming and maintaining alliance relationships. I test the hypothesis males who have the highest consortship rates are also the most 'central' in terms of their within- and between- group centrality. 
&& I test the hypothesis that second-order alliances that have more variation in relationships (higher entropy) exhibit more resilience to change. 
% Complexity - Descriptive
& How complex are 3rd order alliances? 
&& I use the previous analyses to provide a qualitative and quantitative evaluation of the \emph{complexity} of the third-order alliance network, and discuss how the kind of social complexity exhibited by this population may be mediated through social interaction, which I evaluate further in Chapter 3. 
\end{easylist}

\section{Cognitive Complexity \& Behavioral Analysis}
% Interactions are known to mediate relationships and have functional consequences. 
Among primates, extensive research has revealed functional consequences of social interactions, and detailed how differences in social interaction play a role in the mediation of social relationships. Interactions can have immediate consequences, as in unrelated female vervet monkeys for whom grooming prior to a call for help increased affiliates response duration \citep{seyfarth+cheney:1984}, or among male chimpanzees who directed grooming towards potential allies in moments of intense social conflict \citep{nishida:1983}. In other contexts, interactions may be used to test or strengthen bonds between individuals \citep{zahavi:1977}. For example, eye-poking between capuchins \citep{perry:2011} or sexual greetings among adult male baboons \citep{smuts:1990} may be a test of the strength of the relationship between individuals (in the latter study, the strongest male alliance was also the only dyad that showed complete symmetry in their exchange of sexual greetings). There is also evidence that certain kinds of interactions among individuals may mediate reproductive success among social mammals. For example, in adult female savannah baboons, time spent grooming and associating with other adult females is correlated with increased reproductive success \citep{silk:2003}.

This study takes a close look at interactions among allied male bottlenose dolphins to better understand how the alliance association network described in the preceding chapter may be mediated by social interaction. Behavioral observation of alliances \emph{in situ} suggests that \emph{coordinated social behavior} is a fundamental component of mate competition and consortship, two principle activities of alliance formation. During intense contests between second-order alliances, males can be observed coordinating in both space and time as they engage in behaviors such as charging side-by-side, surfacing synchronously as they initiate a contest, or charge opponents. During consortships, allies can be observed coordinating in space and time when they follow behind a consort, produce intricate sequences of mirrored synchronous behavior around a consort, or simply to surface or dive in synchrony. Of such behaviors, only synchrony, which has a reliable surface component, has been quantified due to the inherent challenge of reliably quantifying the many other kinds of behavior that often occur underwater and out of the sight of boat-based observers \citep{connor:2006}. In this study we used aerial videography paired with boat-based data collection to surmount this challenge \citep{nowacek:2001, hodgson:2007, friedman:2014}. 

I predict that dyads will show differences in the type and context of their interactions, and that these differences will be a better predictor of reproductive success than association index. That is, it is not just associating together, but \textit{what allies do when they're together} that makes an alliance. Beyond this, I predict that the kinds of interactions that strong allies use to mediate their relationships should bear parsimonious relationship with the kinds of interactions allies require to succeed in the primary contexts of their existence, namely mate competition and consortships. Allies should \emph{enact} \citep[c.f.][]{VRT:1991} their relationships via the same types of interactions required for reproductive success. 

% Form: synecdoche
The existence of alliances puts males in direct reproductive competition with the same individuals they associate with, and require for success in mate competitions and maintained consortships. If alliances are primarily about reproductive success, then the types of interactions they engage in may communicate something about the current status of their relationship (i.e. allies / competitors), and their probability of success as an alliance, both to the individuals involved in such interactions, and/or in advertisement to others. % Cite: Smolker
%
While explicit examples of such coordinated coalitionary behaviors are rare in the literature, % I could not find ANY examples that captured what I am proposing. maybe birds chorusing?
the widespread use of \emph{derived} or \emph{allochthonous} \emph{intention movements} were described in detail by many of the mid-century ethologists \citep[e.g.][]{daanje:1950,hinde:1952, tinbergen:1952}. For example, \citet{hinde:1952} observed that the "... head-up aggressive posture, which normally appears in reproductive fighting, occurs occasionally in food disputes between Great Tits during the period before the break-up of the flocks." Similarly, threat-postures used by social mammals (canines, primates, felines) such as pilo-erection and bared teeth are known to preempt the escalation of contests. In each case the observed behavior is a \textit{synecdoche}: a physical part of a behavior used by individuals when the nature of the interaction is more intense. 


% Function: assessing the status of relationships = mediating realtionships
The audience of coordinated social behavior relates to the possible function of such interactions. Allies may engage in social coordination as a means of communicating the nature of their relationship to each other, as a mating display to female consorts, or as a display used during intra-sexual competition. 
%
This chapter examines the prediction that allies use coordinated social behavior to mediate their social relationships. I examine two contexts to understand the variation in production of coordinated social behavior: first-order male allies with a female consort, and third-order fusion events occurring between two first-order male consortship groups from different second-order alliances. I test the hypothesis that first-order allies who frequently associate and consort together engage in the highest rates of social coordination, and that third-order allies engage in higher rates of coordinated social behavior and social vocalizations than after 10-15 minutes of association. As a further test of the function of social coordination, I compare rates of social coordination among first-order allies during the consortship context and the two third-order contexts to examine the evidence for social coordination as an inter- or intra- sexual display.

%This work is grounded  in ... 
% Collins, Perry. Silk, Zahavi
%Enacting relationships. Reinforcing and testing bonds. Zahavi, Perry, *Silk. - Already done above? OK?
% Function: advertising the status of relationships

The hypotheses below build consecutively. H1-H2b establish the variation in the production of coordinated social behavior among male allies, and relate that variation to the strength of the alliances. H3 and H4 examine the context of production, providing a more direct test to the prediction that males use social coordination to mediate their alliance relationships. H5 examines the directionality of petting events to test for the presence of asymmetric relationships. 

\begin{easylist}
\ListProperties(Hide2=10,Progressive*=.5cm,Start1=1,Space=2mm)
%\vspace{5mm}
& I look at the context of consortship groups (1st-order males with a female) to understand how variable the production of social coordination (SC) is among male allies, and relate this to the rate at which each dyad associates and consorts together. 
&& H1: There are differences in the production of coordinated behavior among male allies. 
&& H2a: Allies with higher dyadic \textit{association} indices (those who associate together most frequently) also engage in coordinated social behavior more frequently than males with lower dyadic association indices
&& H2b: Allies with higher dyadic \textit{consortship} indices (those who consort together most frequently) also engage in coordinated social behavior more frequently than males with lower dyadic consortship indices
&& H2c: Dyadic consortship index is predicted better by SC rate than by association index. % This is not REALLY reproductive success... 

& Then I look at the context of fusion events between two consortship groups from different 2nd order alliances (3rd-order fusion events) to further understand how allies use social coordination to mediate their relationships. 
&& H3: The production of SC among third-order allies will be higher at fusion events than 10+ min post-fusion. 
&& H4: Third-order fusions will be accompanied by higher rates of acoustic vocalizations (whistling) than 10+ min post-fusion

& To better understand the function of SC, I compare the rates that 1st-order allies engage in coordinated social behavior across the three contexts scored, and test whether rates are significantly higher in any one context. 
\vspace{2mm}

& I look at the directionality of petting events scored in both contexts to address the hypothesis that some individuals initiate more petting bouts than others, and that those that initiate petting bouts consort less frequently than those individuals who they direct petting towards.
&& H5: When the directionality of a petting event can be discerned, males direct petting towards partners who have a higher consortship rate than themselves. 
% H6: Directed petting towards males with higher bStrength? %Explicitly linking to Ch. 2

& To explicitly describe the \textit{process} of coordination, I use a limited set of the scored behaviors (displays or synchrony), to conduct a fine-scale descriptive analysis of the coordination of visible sensorimotor resources through space and time. 
\vspace{2mm}
\end{easylist}

The analyses conducted this chapter will be used to describe the \emph{cognitive complexity} exhibited by the male alliance network in terms of the range of interactions observed, the context and embeddedness of their production, the process of interactions, and how such patterns relate to the structure and dynamics of the web of relationships exhibited by the alliance network. % UNPACK! 


%%%%%%%%%%%%%%%%%%%%%
% REFERENCES % 
%%%%%%%%%%%%%%%%%%%%%
\clearpage
\bibliographystyle{apalike}
\bibliography{/Users/Whitney/Dropbox/Articles/bibliography-wf}

%%%%%%%%%%%%%%%%%%%%%
\end{document}