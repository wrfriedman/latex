\documentclass[11pt]{amsart}
\usepackage{geometry} 
\geometry{letterpaper, margin=1in}
\renewcommand*\rmdefault{ptm} % set to times

\usepackage{natbib}
\usepackage{mathtools}
\usepackage{graphicx}
\usepackage{caption}
\usepackage{subcaption}
\usepackage{wrapfig}
\usepackage{sidecap}
\usepackage{textcomp}
\usepackage{lscape}
\usepackage{framed}
\usepackage{enumitem}
\usepackage[ampersand]{easylist}
\usepackage[font={footnotesize}]{caption}

\usepackage[final]{pdfpages}

\usepackage [english]{babel}
\usepackage [autostyle, english = american]{csquotes}
\MakeOuterQuote{"}

\usepackage{longtable,array}

\graphicspath{
{/Users/Whitney/Dropbox/Aerial/Dissertation/images/}
{/Users/Whitney/Dropbox/Aerial/DataAnalysis/Fusion/}
{/Users/Whitney/Dropbox/Aerial/VideoScoring/FINAL_SCORED/2014-10-07-1127-Images/}}
\title{Chapter 3b: \\ Third-Order Fusions}
\date{}

%%%%%%%%%%%%%%%%
\begin{document}
\thispagestyle{empty}
\pagestyle{empty}
\section*{Table 1: Fusion Event 4A}
% FUZ4: PAS PON QUA PIC PPR	SMO COO URC NIR
% COO is ambassador. Also clearly occurs in 2 other events. FUZ8A: IMP, FUZ10A: PAS. In FUZ 4B QUA lingers longer than the rest.
\noindent 07-October-2014, 09:23:53 (recording time 0:53:59.0). 
Three males from the KS alliance (PAS PON QUA), along with their consorted female (PIC) and her calf (PPR) fuz with three males from the RR alliance (SMO COO URC), along with their consorted female (NIR).

\vspace{3mm}

\noindent Below is a description of the fusion event with five-second sampling frames supplemented when key events fall between the interval sample. Behaviors are described by their corresponding name in the ethogram (Section B, Figure 4). Italics indicate examples of behavioral patterns that are starting to emerge through observation of the aerial video and of the alliances \textit{in situ}, and may be the subjects of future study.

\vspace{3mm}

\noindent \textit{At Fusion}
\begin{longtable}{| m{.3\textwidth} | m{.1\textwidth} | m{.6\textwidth} |}
\hline
Frame & Time & Description \\ \hline
\includegraphics[width=.3\textwidth]{2014-10-07-1127-53-59-0.png} & 53:59.0 &  The KS group, all abreast (ABR), move to fuse head-to-head with COO from the RR group. \textit{COO fuses ahead of the rest of his group, possibly acting as an 'ambassador'} \\ \hline
\includegraphics[width=.3\textwidth]{2014-10-07-1127-54-04-0-Ambassador.png} & 54:04.0 & COO is clearly positioned between the abreast KS group and the rest of the RR group. The abreast RR group follow. PAS and PON are distance-zero. \\ \hline
\includegraphics[width=.3\textwidth]{2014-10-07-1127-54-09-0.png} & 54:09.0 & COO approaches the KS group, heading towards the consorted female of the KS group (PIC) and her calf (PPR). PAS and PON rub (no clear actor/recipient). \\ \hline
\includegraphics[width=.3\textwidth]{2014-10-07-1127-54-12-1.png} & 54:12.1 & COO joins the group diving under, distance-zero, to PPR. The KS are distance-zero and abreast. \textit{Possible use of PPR as a passport by COO.} \\ \hline
\includegraphics[width=.3\textwidth]{2014-10-07-1127-54-14-0.png} & 54:14.0 &  COO joins at the rear of the group, approaching PON then PAS distance-zero. The rest of the RR group lingers outside of 10-meters, not yet fuzed. \textit{Approaching to the rear of the group may precede amicable encounters} \\ \hline
\includegraphics[width=.3\textwidth]{2014-10-07-1127-54-16-4.png} & 54:16.4 &  COO approaches and rubs alongside QUA. PAS PON and COO are all distance-zero. PON approaches PIC to distance-zero. \\ \hline
\includegraphics[width=.3\textwidth]{2014-10-07-1127-54-19-0.png} & 54:19.0 &  COO continues to rub while turning away from QUA, and approaching PIC to distance-zero. PON and PIC are still distance-zero, as are PIC and PPR. The rest of the RR group are still outside of 10-meters but two are abreast behind SMO, approaching the KS group in a group "formation". \textit{COO's interactions with PIC are instances of out-male-approach (OMA).} \\ \hline
\includegraphics[width=.3\textwidth]{2014-10-07-1127-54-24-0.png} & 54:24.0 & COO gooses PPR, orienting his rostrum to the calf's genital region. PPR reacts moving peduncle upwards out of the water. QUA approaches to rub alongside PAS. \textit{COO's interaction with the calf again suggests the possible of use of PPR as a passport.} \\ \hline
\includegraphics[width=.3\textwidth]{2014-10-07-1127-54-29-0.png} & 54:29.0 &  QUA and PPR dive down out of sight. PAS PON PIC are abreast. COO is staggered behind PIC. \textit{COO has returned to his original position closest to the female being consorted by his third-order allies.} \\ \hline
\includegraphics[width=.3\textwidth]{2014-10-07-1127-54-34-0.png} & 54:34.0 &  SMO URC, almost abreast, fuse with KS and COO (now within 10-meters). PON moves over the top of PIC, distance-zero. COO is still distance-zero alongside PIC on the outside of the group. \\ \hline
\includegraphics[width=.3\textwidth]{2014-10-07-1127-54-34-5.png} & 54:34.5 &  PON turns towards COO (all other behavior remains constant from the previous frame). \textit{PON has moved from his position towards COO, possibly to start to move him away from PIC} \\ \hline
\includegraphics[width=.3\textwidth]{2014-10-07-1127-54-39-0.png} & 54:39.0 &  PON moves between PIC and COO. COO is briefly ABR (54:38.1) before beginning to move back towards PIC (54:38.9). The KS and remaining RR continue to move towards a head-to-head fuse, and are now within two body-lengths (distance-2). \emph{PON displaces COO}. \\ \hline
\includegraphics[width=.3\textwidth]{2014-10-07-1127-54-42-0.png} & 54:42.0 & COO moves between PIC and PON. \emph{COO displaces PON} \\ \hline
\includegraphics[width=.3\textwidth]{2014-10-07-1127-54-44-0.png} & 54:44.0 & COO and PAS dive down. The remaining RR and KS close to within one body-length (distance-1). The approach is still head-to-head, with SMO URC abreast, and QUA PIC PON abreast. \\ \hline
\includegraphics[width=.3\textwidth]{2014-10-07-1127-54-46-0-Fusion.png} & 54:46.0 & The remaining RR and KS meet head-to-head, with SMO nearest to QUA and URC nearest to PIC. \\ \hline
\includegraphics[width=.3\textwidth]{2014-10-07-1127-54-46-9.png} & 54:46.9 & URC passes close by PIC in cross-parallel orientation. SMO moves over QUA towards URC and PIC. \textit{URC and SMO drive a separation and converge on the open space closest to PIC. Another examples of OMA}. \\ \hline
\includegraphics[width=.3\textwidth]{2014-10-07-1127-54-49-0.png} & 54:49.0 &  URC and SMO (stacked vertically with SMO above URC) pass through the KS to the rear of the group, distance-0 with PIC. URC and SMO are temporally and spatially coordinated in a "CDC" display around PIC. \\ \hline
\includegraphics[width=.3\textwidth]{2014-10-07-1127-54-54-0.png} & 54:54.0 &  COO emerges again distance-1 and abreast with PIC. PON crosses over COO distance-0 towards the outside of the group. \textit{OMA}. \\ \hline
\includegraphics[width=.3\textwidth]{2014-10-07-1127-54-59-0.png} & 54:59.0 &  NOG emerges, rubbing under COO as he approaches PIC. PPR is in infant-position below PIC. PON remains on the outside of the group. \textit{OMA}. \\ \hline
\includegraphics[width=.3\textwidth]{2014-10-07-1127-55-04-0.png} & 55:04.0 & SMO initiates a ventral-ventral rub (VRB) with QUA, and both males are distance-0 to COO. URC and PON are distance-0. PPR is in infant-position below PIC. \textit{The third-order allies have 'paired off' and are engaging in contact behavior, which may seem affiliative. However, notice that in both dyads, the RR male is closer to PIC than the KS males who have been consorting her. }\\ \hline
\includegraphics[width=.3\textwidth]{2014-10-07-1127-55-09-0.png} & 55:09.0 &  SMO rubs alongside QUA, while QUA approaches and rubs PON. PON and URC are abreast with URC now on the outside of the group. COO is approaching PIC. \textit{OMA}. \\ \hline
\includegraphics[width=.3\textwidth]{2014-10-07-1127-55-14-0.png} & 55:14.0 & PON and PAS orient rostrum-towards to contact URC. PIC and COO are abreast.  SMO QUA and NIR are staggered, with NIR abreast to URC. \\ \hline % DISPLAY NOT CONVINCING.
\includegraphics[width=.3\textwidth]{2014-10-07-1127-55-19-0.png} & 55:19.0 & NIR and URC rub. URC rubs PON with his flukes as PON moves alongside. SMO rubs QUA with his flukes as QUA moves alongside. \textit{Third-order contact behavior} \\ \hline
\includegraphics[width=.3\textwidth]{2014-10-07-1127-55-24-0.png} & 55:24.0 & NIR rubs under URC.  PON and URC are still distance-0. SMO and QUA are abreast. \textit{Third-order contact and formation.} \\ \hline
\includegraphics[width=.3\textwidth]{2014-10-07-1127-55-29-0.png} & 55:29.0 & URC and NIR remain at the surface, almost abreast and almost in contact. \\ \hline
\includegraphics[width=.3\textwidth]{2014-10-07-1127-55-34-0.png} & 55:34.0 & URC and NIR dive down distance-zero, asynchronously.  \\ \hline
\end{longtable}

\noindent \textit{5-minutes Post-Fusion}
\begin{longtable}{| m{.3\textwidth} | m{.1\textwidth} | m{.6\textwidth} |}
\hline
Frame & Time & Description \\ \hline
\includegraphics[width=.3\textwidth]{2014-10-07-1127-59-24-0.png} & 59:24.0 &  PPR is in infant-position under PIC. PAS and PON are the closest males to PIC and are distance-0. Others are emerging. \\ \hline
\includegraphics[width=.3\textwidth]{2014-10-07-1127-59-29-0.png} & 59:29.0 & The group has become mostly parallel and abreast. PPR is in infant-position under PIC. PAS and PON are abreast. QUA and URC are abreast. SMO reaches out with his flukes to rub URC. COO and NIR are abreast. \\ \hline
\includegraphics[width=.3\textwidth]{2014-10-07-1127-59-34-0.png} & 59-34-0 & PAS QUA NIR are abreast. PPR is in baby-position next to PIC  \\ \hline
\includegraphics[width=.3\textwidth]{2014-10-07-1127-59-39-0.png} & 59-39-0 &  PAS PON QUA are abreast. PPR is in infant-position below PIC Others are below QUA with some white flashes indicating ventral turns but individuals can't be distinguished or scored. \\ \hline
\includegraphics[width=.3\textwidth]{2014-10-07-1127-59-44-0.png} & 59-44-0 & PIC and NIR are on the outer edges of the group. PPR is in infant-position below PIC. PAS and PON are abreast. SMO and COO do a spatially and temporally coordinated side-press which starts out around QUA and URC. \\ \hline
\includegraphics[width=.3\textwidth]{2014-10-07-1127-59-46-7.png} & 59:46.7 & SMO and COO continue the coordinated side-press display now just around URC. \\ \hline
\includegraphics[width=.3\textwidth]{2014-10-07-1127-59-49-0.png} & 59:49.0 & SMO and COO finish the coordinated side-press display around URC. QUA and NIR are abreast in formation (FRM) behind URC. QUA and URC are distance-0. PAS and PON are abreast. PIC and NIR are still on the outside of the group, with PPR in infant-position below PIC. \\ \hline
\includegraphics[width=.3\textwidth]{2014-10-07-1127-59-54-0.png} & 59:54.0 & PIC is still on the outside of the group with PAS PON closest and abreast. NIR does a ventral-rub (VRB) to URC.  \\ \hline
\includegraphics[width=.3\textwidth]{2014-10-07-1127-59-59-0.png} & 59:59.0 & NIR continues to rub under URC. QUA and URC are abreast. \\ \hline
\includegraphics[width=.3\textwidth]{2014-10-07-1127-1-00-04-0.png} & 1:00:04.0 & QUA rubs URC. \textit{Lots of attention is being paid to URC in this bout. Why?} \\ \hline
\includegraphics[width=.3\textwidth]{2014-10-07-1127-1-00-09-0.png} & 1:00:09.0 &  QUA and URC engage in mutual pec-to-pec petting. \\ \hline
\includegraphics[width=.3\textwidth]{2014-10-07-1127-1-00-14-0.png} & 1:00:14.0 &  PIC PPR PAS PON surface again, abreast, with PAS PON between PIC and the rest of the group. PIC and NIR are on the outer edges of the group again, with PPR in baby-position next to PIC. QUA rubs under URC. Groups are starting to separate, but QUA continues to interact with the RR group. \textit{QUA remains interacting affiliatively with the RR while PAS PON maintain a position separating PIC from the RR. Simultaneously offensive and affilitiave roles within the same group. QUA's position is similar to that of COO's in the beginning, marking QUA as a potential 'ambassador'} \\ \hline
\includegraphics[width=.3\textwidth]{2014-10-07-1127-1-00-19-0.png} & 1:00:19.0 & PIC PAS PON are abreast with PIC on the outer edge of the group, and all three trailing behind the RR group. QUA rubs under URC before turning away to return to his group. \\ \hline
\includegraphics[width=.3\textwidth]{2014-10-07-1127-1-00-24-0.png} & 1:00:24.0 &  PIC PAS PON are abreast trailing the RR. PPR is in infant-position beneath PIC. QUA is returning to his group, and the RR are diving. The KS and RR continue to separate. After this, SMO dives and the next time the RR group surfaces they have fizzed / are greater than 10-meters from KS. \\ \hline

\end{longtable}
%%%%%%%%%%%%%%%%%
\end{document}
