\documentclass[11pt]{amsart}
\usepackage{geometry} 
\geometry{letterpaper, margin=1in}
\renewcommand*\rmdefault{ptm} % set to times

\usepackage{natbib}
\usepackage{mathtools}
\usepackage{graphicx}
\usepackage{caption}
\usepackage{subcaption}
\usepackage{wrapfig}
\usepackage{sidecap}
\usepackage{textcomp}
\usepackage{lscape}
\usepackage{framed}
\usepackage{enumitem}
\usepackage[ampersand]{easylist}
\usepackage[font={footnotesize}]{caption}

\usepackage[final]{pdfpages}

\usepackage [english]{babel}
\usepackage [autostyle, english = american]{csquotes}
\MakeOuterQuote{"}

\graphicspath{
{/Users/Whitney/Dropbox/Aerial/Dissertation/images/}
{/Users/Whitney/Dropbox/Aerial/DataAnalysis/Fusion/}
{/Users/Whitney/Dropbox/Aerial/VideoScoring/FINAL_SCORED/2014-10-07-1127-Images/}}
\title{Chapter 3b: \\ Third-Order Fusions}
\date{}
%%% BEGIN DOCUMENT
\begin{document}

\maketitle

%%%%%%%%%%%%%%%%%%%%%
\section{Introduction}
%%%%%%%%%%%%%%%%%%%%%
In this chapter I examine the context of two first-order male trios with female consorts from different second-order alliances joining together ("third-order fusions") to examine how third-order alliances are mediated through social interaction. I test the hypothesis that males participate in social coordination at higher rates at the time of fusion than later periods during their time together. Additionally, I describe the difference in interaction rates between second- and third-order allies, as well as differences pre- and post- fusion in female- directed behaviors. Detailed descriptive analyses are provided for a limited set of fusion events.

% 2nd vs. 3rd
% 3rd order @ / post fusion
% 2nd order @ / post fusion
% OMA
\vspace{3mm}
\noindent I examine the context of fusion (joining) events between two consortship groups from different 2nd order alliances (3rd-order fusion events) to further understand how allies use social coordination to mediate their relationships. 
\begin{itemize}
\item H5: The production of SCC among third-order allies will be higher at fusion events than 5+ min post-fusion. 
\item H4: Third-order fusions will be accompanied by higher rates of acoustic vocalizations (whistling) than 5+ min post-fusion
\end{itemize}

\noindent To explicitly describe the process of fusion interactions, I use a limited set of the scored behaviors (displays or synchrony), to conduct a fine-scale descriptive analysis of the coordination of visible sensorimotor resources through space and time. 

\subsection{Background}
\vspace{3mm}
% BACKGROUND: Why would we expect males to increase these kinds of events? To show unity, enact relationships. Reinforce / remind relationships. Establish today's context.
*

* Fission-fusion dynamics in animal societies \citep{aureli:2008, connor:2007,smolker:1992}

* Complexity, cognitive burdens of fission-fusion \citep{connor:2007, connor:2010}

* Why expect males to show increased social coordination at fusions?

* Establishing, mediating, enacting relationships

* Preceding (mediating? communicating?) affiliative / agonistic encounters

%%%%%%%%%%%%%%%%%%%%%
\section{Methods}
%%%%%%%%%%%%%%%%%%%%%
\subsection{Subjects}
The subjects for this analysis were groups containing two first-order male trios from two different second-order alliances and at least one female consort. All males were in one of the three focal second-order alliances (KS, PD, or RR). 
\footnote{Since a first-order alliance is a subset of a second-order alliance, but no additional second-order allies were included in this analysis, in this chapter "first-order allies" are synonymous with "second-order allies". "Third-order allies" refers to groups comprised of males from two different second-order alliances.} % choose one and stick with it

\subsection{Fusion Events}
From 2013-2014, X fusion events were recorded during our focal follows. % 398 fusion events total in entered focal scans!
26 events % This is how many ended up in the "events" spreadsheet. Would have to look back to see how many that was whittled down from.
were between two trios from different second-order alliances with at least one consorted female. Of these, there were 10 fusion events that occurred on different days and/or between different subgroups, were the first fusion events that were recorded between these groups for the day, had a known time of fusion, and were recorded on aerial video in winds less than 12 mph (Beaufort 3 or less). 
\footnote{There is some evidence to suggest that first-fusions may have more activity than fusions occurring after only brief separations.}
An additional 3 fusion events met all the same criteria except were not the first fusions of the day. These events were sampled during the "post-fusion" time period only. The 13 events used in this analysis varied in length from 2min57s to 37min59s (mean = 8min54s, sd = 9min52s).

\vspace{3mm} *ADD: Number of events per alliance / individual ... \vspace{3mm} %ATTN%

Fusion events were scored from 0-5 minutes ("at fusion") and from 5-13 minutes ("post-fusion"). Scoring stopped two minutes prior to a group fission or the fusion of another group. Since fusion events varied in length, events were scored for as many intervals possible, up to the length of time indicated. Events are described in terms of rates of activity over the set of intervals possible to score for each event (\S\ref{sec:analysis}). The number of intervals scored was also limited by the presence of focal individuals near the surface and visible for observation on the aerial video.

\subsection{Behavioral Interactions}
\subsubsection{Sampling}
Aerial and side-angle video data were sampled using the five-second group instantaneous sampling method described in \S{B.2.5}. Sampling began at the first evidence of fusion, defined as either (A) the first frame of aerial video in which at least one individual from each of the two subgroups were visible for observation and within 10-meters, or (B) if the fusion was not captured on the aerial video, the exact time that observers identified the fusion event. A "fuzed" group was defined by a "10-meter-chain rule" \citep{smolker:1992} where individuals were considered part of the focal group as long as they were within 10-meters of another dolphin. Distance was estimated by body-lengths, where one-body length is estimated to be approximately 2-meters, and was measured on the aerial video to the nearest 0.2cm when in question.

% ALL THIS IN SECTION 2.5 of 3-BehavioralMediation-Intro!
% At each interval, the behavior of each dyad or polyadic group was assessed in the categories ...
%Sync, pet, contact, display, agonism
%Actors and recipients

\subsubsection{Analysis}\label{sec:analysis}
Each five-second point sample contained the identities (or unknown codes when identities could not be verified) of all individuals visible for observation, as well as their on-going behavior in the categories of contact, synchrony, displays, formations, and agonism (see detailed ethogram in \S{B: Figure 4} %FIX%
From these point samples, \emph{Interaction Rates (IR)} were calculated as follows for each sampling period: 

\begin{equation} \label{eq:1}
IR = \frac{\text{nDyads in Behavioral Category ($\beta$) in Period }\tau}{\text{nDyads Present for Observation ($\Theta$) in Period }\tau}  = \frac{\sum\limits_{i \in \tau} \beta{Dyads}}{\sum\limits_{i \in \tau} \Theta{Dyads}},
 \end{equation}

Where $\tau$ refers to the complete set of 5s intervals in a sampling period at fusion (minutes 0-4.99) or post-fusion (minutes 5-12.99), and \emph{i} refers to a single interval within the set $\tau$. Only one period at- and post- fusion was sampled per third-order group, per day. If multiple fusion events between the same subgroups were captured in a day, only the first fusion event captured on that day was sampled. Because fusion events were of differing lengths, and because not all events were captured on aerial video for the entire length of the interaction, it was not possible to construct paired samples for each event (i.e. not all "at-fusions" have a paired "post-fusion" sample, and vice versa).

Equation \ref{eq:1} was modified for interaction rates among second-order allies (Eqn. \ref{eq:2}) or thrid-order allies (Eqn. \ref{eq:3}) as follows: 

\begin{equation} \label{eq:2}
IR_{2} = \frac{\sum\limits_{i \in \tau} \beta_{2}{Dyads}}{\sum\limits_{i \in \tau} \Theta_{2}{Dyads}},
\end{equation}

\begin{equation} \label{eq:3}
IR_{3} = \frac{\sum\limits_{i \in \tau} \beta_{3}{Dyads}}{\sum\limits_{i \in \tau} \Theta_{3}{Dyads}},
\end{equation}

Where $\beta_{2}{Dyads}$ is the number of second-order dyads and $\beta_{3}{Dyads}$ is the number of third-order dyads in behavioral category $\beta$, and where $\Theta_{2}{Dyads}$ is the number of second-order dyads and $\Theta_{3}{Dyads}$ is the number of third-order dyads visible for observation in the interval sample set $\tau$.

% Additional Behavioral Categories - ADD: WHY THESE CATEGORIES? SOMEWHERE
In addition to the sampled behavioral categories, data were re-classified into the categories following for further analysis. Each pair was counted only once per sampling interval.

\begin{itemize}
\item Any Interaction (AnyIxn): male pairs engaged in any type of interaction (contact, displays, formations, synchrony, or agonism). Pairs could be both actors, both recipients, or actor-recipient pairs.
\item Any Coordination (AnyCoord): male pairs engaged in any spatially and/or temporally coordinated activity (display, formation, synchrony, or agonism), where both males are actors. 
\item Coordinated Male-Male Agonism (AgPairs): male pairs engaged in potentially agonistic activities (display, formation, agonism) where the recipient is another male. Both males are actors.
\item Actor-Recipient Agonism (AgAR): male pairs engaged in potentially agonistic activities (display, formation, agonism) where one male from the pair is an actor and the other is a recipient.
\item Coordinated together (CoordTog): male pairs engaged in coordinated activity (contact, displays, formations, and synchrony) without a recipient. 
\item Coordinated around (CoordAround): male pairs engaged in coordinated activity (contact, displays, formations, and agonism) directed towards or around a recipient.
\item Out-male-approach (OMA): Male-female pairs where the male is actor and female is the recipient in contact, displays, formations, or agonism, or both are actors in synchrony. The female is the consorted female of the third-order male allies of the male actor (not the female that he is consorting with his first-order allies). Equation \ref{eq:oma} is used for this category. 
\item Male-to-Consorted Female (CSO): Male-female pairs where the male is actor and female is the recipient of contact, displays, formations, or agonism, and both are actors in synchrony. The female is the same female that he is consorting with his first-order allies. Equation \ref{eq:cso} is used for this category. 
\end{itemize}

% each male pair only counted once per interval

\begin{equation} \label{eq:oma}
IR_{oma} = \frac{\sum\limits_{x \in \tau} \beta_{oma}{Dyads}}{\sum\limits_{x \in \tau} \Theta_{oma}{Dyads}},
\end{equation}

Where $\beta_{oma}{Dyads}$ is the number of male-female OMA pairs engaging in OMA behavior, and where $\Theta_{oma}{Dyads}$ is the number of possible OMA dyads visible for observation in the interval sample set $\tau$.

\begin{equation} \label{eq:cso}
IR_{cso} = \frac{\sum\limits_{x \in \tau} \beta_{cso}{Dyads}}{\sum\limits_{x \in \tau} \Theta_{cso}{Dyads}},
\end{equation}

Where $\beta_{cso}{Dyads}$ is the number of male-female pairs engaging in CSO behavior, and where $\Theta_{cso}{Dyads}$ is the number of possible CSO dyads visible for observation in the interval sample set $\tau$.

% Statistics in R{pkg}

\subsection{Acoustic Production}

Acoustic production was sampled over the same periods as third-order fusions and post-fusions. % ENSURING THAT THE DOLPHINS WERE NEAR ENOUGH TO THE BOAT TO HAVE THEIR ACOUSTIC PRODUCTIONS CAPTURED
A one-second interval sampling method was employed to capture the rate of acoustic production during these contexts. Acoustic data were reviewed using spectrographic displays (Fast Fourier Transform size: 1024, weighting function: Hanning window, time resolution: 3s, sampling rate: 192 kHz) using the free, open-source software Audacity. On each one-second interval, the observer recorded the presence of whistles ("W"), burst-pulse calls ("B"), pops ("P"), or echolocation ("E"). 

%Those recording segments in which engine noise exceeded 2 kHz were discarded from the analysis. King & Janik 2014

% Whistles have a contour and are continuous through time. 
% Burst-pulse calls are punctuated - if a series was on-going at the 1-second interval, was called "present" regardless of whether that exact point in spectral space indicated signal...
% Produce / see figures... 

\subsection{Descriptive Analysis}
% FUZ1: DNG NOG IMP SUR SON	SMO URC COO DEM
% 20-Nov-13	4a	9:42	NOG

% FUZ4: PAS PON QUA PIC PPR	SMO COO URC NIR
% COO is ambassador. Also clearly occurs in 2 other events. FUZ8A: IMP, FUZ10A: PAS. In FUZ 4B QUA lingers longer than the rest.

A fine-scale descriptive analysis was conducted for the third-order fusion event on 07-October-2014, 09:23:53. During this event, three males from the KS alliance (PAS PON QUA), along with their consorted female (PIC) and her calf (PPR) fused with three males from the RR alliance (SMO COO URC), along with their consorted female (NIR). The descriptive analysis includes the five-second sampling frames supplemented by key events falling between the interval sample. Behaviors were described by their corresponding name in the ethogram (Section B, Figure 4). Also noted are examples of behavioral patterns that are starting to emerge through observation of the aerial video and of the alliances \textit{in situ}, and may be the subjects of future study.

* A descriptive analysis of a second event is underway. It will not take long to complete.

%%%%%%%%%%%%%%%%%%%%%
\section{Results}
%%%%%%%%%%%%%%%%%%%%%
\subsection{Behavioral Interactions}
\subsubsection{Second vs. Third Order Allies}
%AnyIxn / AnyCoord
Before noting \emph{how} third order allies use interactions to mediate their social relationships, it is useful to first note that they \emph{do} interact with one another, though at a reduced rate when compared to second-order allies. Figure \ref{fig:anyIxn}A shows the interaction rates for second- and third- order male dyads engaged in any type of interaction (AnyIxn), where males were either actor or recipient in any of the behavioral categories sampled. Figure \ref{fig:anyIxn}B shows the interaction rates for second- and third- order male dyads engaged in any type of coordinated interaction (AnyCoord), where both males were actors in the behavioral categories display, formation, agonism, or synchrony. Second-order allies interacted at a higher rate than third-order allies for both AnyIxn (Welch two-sample t(15.9) = 3.74, p$<$.001), and AnyCoord (Welch two-sample t(13.9) = 4.34, p$<$.001).
% Each dyad was only counted ONCE per interval

In particular, I found that second-order allies engage in significantly higher rates of display (Welch two-sample t(15.3) = 1.87, p$<$.05) and formations (Welch two-sample t(14.3) = 4.27, p$<$.001) than third-order allies (Figure \ref{fig:AllEvents}). Second-order allies also engaged in significantly higher rates of CoordTog (Welch two-sample t(14.1) = 3.49, p$<$.01) and CoordAround (Welch two-sample t(14.3) =  2.13, p$<$.05), with both second- and third-order allies engaging in less CoordAround than CoordTog (Figure \ref{fig:AllEvents}). 

\subsubsection{At vs. Post Fusion (H5)}
% Third-order allies
The primary hypothesis I stated at the beginning of this chapter is that third-order allies engage in higher rates of social coordination at fusion events compared to 5-plus minutes post-fusion. In fact the two general indicators, AnyIxn and AnyCoord, indicate the opposite relationship, but do not show significant differences between contexts (Figure \ref{fig:anyIxn3}). However, a closer look at the behavioral categories shows that certain kinds of behavior occurs at a higher rate at fusion, while other kinds of behavior occur at a higher rate post-fusion. In particular, third-order allies engage in significantly more contact behavior (Welch two-sample t(13.9) =  1.86, p$<$.05) and coordinated male-male agonism (Welch two-sample t(10) =  1.83, p$<$.05) at fusion. Post-fusion, third-order allies tended to engage in higher rates of synchrony and formation (Fig \ref{fig:behRates3}). 
% Declining trend. Say more? 
A moving window average (window size = 60s) over the fusion data shows a declining trend in rates of contact among third-order allies (Fig \ref{fig:Con3}).
%OMA
Out-male-approaching behavior (OMA) was seen at significantly higher rates post-fusion than at fusion (Welch two-sample t(6.96) =  -2.33, p$<$.05) (Fig \ref{fig:oma}). 

% Second-order allies
I also tested the hypothesis that second-order allies engage in higher rates of social coordination at vs. post fusion. Rates of AnyIxn and AnyCoord among second-order allies were similar across contexts, with no significant differences. However, second-order allies did show slightly increased rates of AnyCoord at fusion (Figure \ref{fig:anyIxn2}). Again, the behavioral categories showed a more complex pattern: second-order allies engaged in significantly higher rates of displays (Welch two-sample t(11.4) =  2.40, p$<$.05), and tended to show higher rates of Formation and CoordTog at fusions.
% FRM, SS, Displays, around their female consort

%CSO: NS difference pre/post, increased post-fusion.

\subsection{Acoustic Production At vs. Post Fusion (H6)}

*This analysis is underway. It will not take long to complete.

\subsection{Descriptive Analysis}

A full descriptive analysis can be found in Table \ref{fuzDescriptive}. The event begins when three males from the KS alliance (PAS PON QUA), along with their consorted female (PIC) and her calf (PPR) fuse with three males from the RR alliance (SMO COO URC), along with their consorted female (NIR). The groups fuse along a head-to-head trajectory, with individuals from both alliance engaging in an side-by-side abreast formation (ABR). One male (COO) precedes the rest of the RR group, positioned first mid-way between then groups, then joining singly before the rest of his group joined. This behavior, tentatively labelled "ambassador" was observed in two additional fusion events of the 10 scored "at fusion". COO changed from a head-to-head trajectory to finally fuse with the group along the outside, closest to the consorted female of the KS males (PIC) and her calf (PPR). The pattern of the "out-male" approaching or interacting with the consorted female of another group was described earlier in this chapter (OMA); I found that OMA occurred at higher rates post-fusion than at-fusion. COO continued his approach diving under at distance-0 to PPR. This was not an overtly agonistic or affiliative kind of contact, but rather suggests the possible use of calves as "buffers" or "passports" to decrease the probability of agonism among males \citep[e.g.][]{itani:1959, hrdy:1976, silk:1984}. COO continues around the rear of the group, a possible sign of "respect" (R. Connor, pers com), to initiate a rub alongside QUA, before re-approaching PIC and PPR. COO "gooses" PPR, orienting his rostrum towards the calf's peduncle / genital region, and moves again towards the outside of the group nearest PIC and PPR. When PPR dives COO is closest to PIC but appears to be displaced by PON who moves between COO and PIC, briefly, until PON is displaced by COO moving between PON and PIC. This kind of "low-level agonism" among male allies may be a mechanism for establishing relationships, possibly even dominance, without jeopardizing the critical relationship as second or third-order allies. When SMO and URC join, the approach the opposite side of the group, moving together between QUA and PIC and then in unison behind the group. The males form pairs of third-order allies, engaging in contact behaviors rubbing sides, ventrally, and fluke-to-body. This is the kind of pairing was seen in X of the 10 "at-fusion" periods scored. Two KS males, PON and POS approach URC, and NIR rubs under URC. In the post-fusion bout documented, URC continues to receive a lot of attention. Why?

Five-minutes post-fusion, the groups have settled into a mostly abreast orientation relatively to one-another, but there are still third-order interactions. PAS and PON are positioned between the PIC PPR pair and the rest of the KS-RR group. PPR stays in infant-position except to come up to baby-position for a breath. QUA is actively engaged with the RR. He moves above the group as they dive, and when they come up SMO and COO do a spatially and temporally coordinated side-press which starts out around QUA and URC, but as the display continues URC remains at the center and QUA falls behind. Similar to the event described in the "at-fusion" bout, NIR again moves towards URC to rub underneath. QUA then initiates a rub to URC, which turns into mutual pec-to-pec petting between the pair. As the groups begin to separate, QUA remains with the RR a bit longer before rubbing under URC as he returns to his group. The RR dive and when they surface again the two groups are greater than 10 meters apart and have "fizzed".

%%%%%%%%%%%%%%%%%%%%%
\section{Discussion}
* Pre and Post fusion summary 

* Ambassadors: individuals who lead into, and interact more with, 3' allies

* Shouldering, bumping, displacing: Low-level agonism rather than overt aggression (tail-hits, charges, body slams seen in fights) to mediate relationships (dominance?) but not jeopardize critically important alliance relationships? \citep{connor:2001}

* Vying for access to female? Positioning between OMA and consorted female. But OMA increases post-fusion!

* Triadic Interactions \citep{silk:1984}... 

* CoordAround, Coord MM Agonism =  Joint attention \citep{xitco:2001},  bruner, tomasello... 
%Bruner JS (1977) Early social interaction and language acquisition. In: Schaffer HR (ed) Studies in mother-infant interaction. Academic Press, London, pp 271?289
* Dyadic vs. polyadic behavior \citep{kummer:1974} ... 

* Fission-fusion and complexity \citep{aureli:2008}

%%%%%%%%%%%%%%%%%%%%%
% Figures
%%%%%%%%%%%%%%%%%%%%%
\section{Figures}
%%%%%%%%%%%%%%%%%%%%%

\begin{figure}
  \centering
    \includegraphics[width=.7\textwidth]{2nd_vs_3rd_AnyCoordIxn}
  \caption{Rates of AnyIxn(A) and AnyCoord (B) recorded among second-order allies and third-order allies from 0-13min after fusion. Second-order allies engage in significantly higher rates of both categories.}
  \label{fig:anyIxn}
\end{figure}

\begin{figure}
  \centering
    \includegraphics[width=1\textwidth]{2nd_vs_3rd_AllFuzEvents}
	\caption{Behavioral rates (A-H) recorded among second-order allies and third-order allies from 0-13min after fusion.}
  \label{fig:AllEvents}
\end{figure}

\begin{figure}
  \centering
    \includegraphics[width=.7\textwidth]{At_vs_Post_Rates_AnyIxn3}
  \caption{} % elaborate
  \label{fig:anyIxn3}
\end{figure}

\begin{figure}
  \centering
    \includegraphics[width=1\textwidth]{At_vs_Post_Rates3}
  \caption{} % elaborate
  \label{fig:behRates3}
\end{figure}

\begin{figure}
  \centering
    \includegraphics[width=1\textwidth]{MovingWindow_Con3}
  \caption{} % elaborate
  \label{fig:Con3}
\end{figure}

\begin{figure}
  \centering
    \includegraphics[width=.7\textwidth]{At_vs_Post_Rates_AnyIxn2}
  \caption{} % elaborate
  \label{fig:anyIxn2}
\end{figure}

\begin{figure}
  \centering
    \includegraphics[width=1\textwidth]{At_vs_Post_Rates2}
  \caption{} % elaborate
  \label{fig:behRates3}
\end{figure}

\begin{figure}
  \centering
    \includegraphics[width=.7\textwidth]{OMA_CSO_Rates}
  \caption{} % elaborate
  \label{fig:oma_cso}
\end{figure}
% Appendix: python scripts? 

%%%%%%%%%%%%%%%%%%%%%
% Descriptive Analysis %
%%%%%%%%%%%%%%%%%%%%%

\clearpage
\includepdf[pages=-,pagecommand={\thispagestyle{plain}}]{3b-BehavioralMediation-3FUZ-Descriptive.pdf}

%%%%%%%%%%%%%%%%%%%%%
% REFERENCES % 
%%%%%%%%%%%%%%%%%%%%%
\clearpage
\bibliographystyle{apalike}
\bibliography{/Users/Whitney/Dropbox/Articles/bibliography-wf}

%%%%%%%%%%%%%%%%%%%%%
\end{document}