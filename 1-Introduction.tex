\documentclass[11pt]{amsart}
\usepackage{geometry} 
\geometry{letterpaper, margin=1in}
\renewcommand*\rmdefault{ptm} % set to times

\usepackage{natbib}
\usepackage{mathtools}
\usepackage{graphicx}
\usepackage{caption}
\usepackage{subcaption}
\usepackage{wrapfig}
\usepackage{sidecap}
\usepackage{textcomp}
\usepackage{lscape}
\usepackage{framed}
\usepackage{enumitem}
\usepackage[ampersand]{easylist}

\usepackage [english]{babel}

\graphicspath{
{/Users/Whitney/Dropbox/Aerial/Dissertation/images/}}

\title{Chapter 1: \\ Introduction}
\date{}
%%% BEGIN DOCUMENT
\begin{document}

\maketitle

%%%%%%%%%%%%%%%%%%%%%
\section{Introduction}
%%%%%%%%%%%%%%%%%%%%%\begin{quote}

%%%%%
\begin{quote}
September 23, 2010

We departed heading north through the bay, encountering an adult male trio, a \textit{first-order alliance} from the larger \textit{2nd-order} PD alliance in the first channel. The trio, PRI WAB and NAT, were without the female Phantom (or her weaning calf Opera), who they had been consorting just a few days prior.

Continuing north we surveyed a first-order alliance from the the 2nd-order WC alliance, Wow, Pik, and Myrtle, who were consorting the female Stetson. Within 10 minutes they were joined by another first-order trio from the WC alliance, Latch, Gripe and Vee, who were consorting Phantom. 

The two first-order WC alliances and female consorts travelled slowly northwest in two tight subgroups, the two consortship groups slightly separated from one another. During that hour the group maintained a low energy bout, slowly traveling and resting together, sometimes one breaking off to forage while another male stayed close by the female in his group. The two first-order WC alliances never strayed more than 100m from each other. 

Forty minutes after we began our survey another large group appeared on the horizon, moving rapidly toward the WC. Shoulder-to-shoulder, they moved in unison creating a white cresting wave as they charged toward the WC. The two groups met in a wash of splashing, flukes flying. Immediately, a seventh member of the WC who we had not seen for the first forty minutes of observation joined the group. From the surface, we observed a mass of dolphins churning in towards each other, surging, charging, and chasing at times. Quick fin jerks at the surface indicated aggression below. Loud exhales called "chuffs" could be heard, and groups numbering two to five dove in synchrony. 

The incoming group was a third-order alliance comprised of 6 adult males from the 2nd-order AM alliance and 5 adult males from the 2nd-order HG alliance.

After five minutes of intense fighting the groups split briefly - Phantom was still with the WC. Another minute of intense fighting - surging, tail hitting, and aggression and the groups split again. Some of the WC paused at the surface in a tight group, but the third-order AM-HG alliance were still in a excited churning flurry, with WC males Latch and Gripe among them. The WC male Wow left the snagging group, charged into the fight, and surged with Latch and Gripe behind the AM male Orion. Someone was bodily pushed up to the surface from below, and we could hear loud whistles in the air. The group continued surging, meeting head-to-head, with sub-groups surfacing in unison for another five minutes. 12 minutes after the AM-HG arrival, the WC separated with Stetson and Phantom in a tight pack, maintaining distance from the AM-HG. Screams, chuffs, and shrill chirps from the WC group were loud enough to be heard in the air. Immediately following this aural display, Latch returned to the snagging WC, then charged again to re-joining the AM-HG group, trailing at first behind an incredible surge. 

Twenty-six minutes into the fight a new group appeared on the horizon - the AM-HG group moved towards them but no join was observed. These were offshore dolphins not well habituated to the research boat, but had been seen frequently enough with AM that we were starting to suspect a second- or third-order alliance between these "friends of AM" and AM. A few minutes later the incoming group disappeared completely without joining the competing group.

% Add ARN sneaking off to check on female? 

Intense fighting ensued for another 6 minutes - huge group surges accompanied tail hits, whistles in air, and a chase. Phantom and her calf Opera were now amongst the fighting males. The mass of dolphins surged in unison again as another six males from an in-shore 2nd-order alliance (KS) approached the fighting group. It is unclear clear whether the six KS ever got closer than 10 meters to AM/HG, but coincident with their arrival (37 minutes after the arrival of AM-HG) the WC-KS and AM-HG groups split again, this time with AM-HG porpoising and leaping rapidly south - away from WC-KS. The KS excitedly socialized amongst themselves, but were no longer with the WC.

A hundred meters away the WC were traveling slowly in a tight pack. All 7 males and Stetson were present, but Phantom and Opera were not. The group moved slowly beside the boat and we observed petting between the males. We documented and verified the group membership before changing gears to chase down the still leaping AM-HG group. The group was excited when we reached them again: % add detail. Even the (apparently male) calf OPE had an erection. 
As the group settled it became clear that three AM males had begun a new consortship with Phantom. % HOW?
% After trailing behind the AM for ... ? 
The 5 HG males eventually departed, leaving the six male AMs in two subgroups, each consorting a female.
\end{quote}


\section{Introduction}

% Introduction: terms, social and cognitive complexity
The study of the co-evolution of social and cognitive complexity was born in the 1960's and 70's with the Social Intelligence Hypothesis, the proposal that intelligence, rather than evolving in the context of increased foraging or technical skills, was linked in evolutionary ratchet with increasingly complex societies \citep{jolly:1966, humphrey:1976}. Since then research focusing on this topic has been conducted mostly on primates, and for a significant period of time morphed into research on the existence and origins of Machiavellian intelligence \citep{byrne:1988,byrne:1997}. The primate model provides important data in the study of human evolution using homology and divergency; traits we share with close relatives and those uniquely human. But there is another approach. The study of convergent evolution explores the independent evolution of similar traits in more distantly related taxa; an approach that can help us understand the context and mechanisms which may facilitate the evolution of similar biological phenomena. To further understand the co-evolution of complex societies and complex cognition, there may be utility in looking beyond non-human primates to a taxa where social structures may be more similar to our own than any extant primate.

Male bottlenose dolphins in Shark Bay, Western Australia participate in up to three levels or \textit{orders} of alliance formation within a large open society of resident dolphins. \textit{First-order} alliances are comprised of pairs and trios of adult males who work together to sequester and consort estrus females \citep{connor:1992a}. Most adult males also belong to teams of 4-14 individuals, or \textit{second-order} alliances, who work together to take females from other males and to defend against such attacks \citep{connor:1992a, connor:1992b}. In 2011, a \textit{third-order} of alliance formation was described in which two second-order alliances that exhibit low-level but consistent associations coordinated against other males during competitions over females \citep{connor:2011}. 

Such nested structure where interactions among the same components at different scales result in functional differences meets the same criteria for \textit{complexity} used to describe the neural networks that constitute the human brain \citep{tononi:1994}. The nested structure of alliances in Shark Bay may be the most complex outside of humans \citep{connor:2007}, but while much work has been done to describe the function of first- and second-order alliances, to date little is known about third-order alliances, how the structure of alliance networks are maintained, or how they respond to change.

In R.A. Hinde's classic scheme, a population's social structure is described by its pattern of social relationships, which, in turn, are comprised of a history of interactions among individuals \citep{hinde:1976}. In this dissertation I traverse Hinde's levels to examine the relationship between social complexity through quantitative analysis of a third-order alliance network, and social cognition through fine-scale observations of behavioral interactions among allies. I aim to make two kinds of contributions (A) to lay a few boards in the growing methodological bridge towards better understanding the relationship between, and co-evolution of, complex societies and complex cognition, and (B) to further our understanding of dolphin society and cognition. There are two principal questions that guide the research in this dissertation: (1) how are third-order alliances structured, what are their dynamics, and how do they respond to change? (2) How do the particular kinds of behavioral interactions that allies engage in relate to the observed structure and dynamics of the alliance network?

\begin{figure}[h]
  \centering
    \includegraphics[width=.8\textwidth]{thesis-in-a-nutshell}
  \caption{The theory, methods, and big picture questions discussed in this dissertation}
  \label{fig:dissertationNutshell}
\end{figure}

\subsection{Social and Cognitive Complexity}

In the 1960's and 70's, advocates of the \textit{Social Intelligence Hypothesis} proposed that intelligence, rather than % In addition to?
evolving in the context of increased foraging or technical skills, was linked in evolutionary ratchet with increasingly complex societies \citep{jolly:1966,humphrey:1976}. Jolly wrote, ''since [social primates'] dependence on the troop both demands social learning and makes it possible, social integration and intelligence probably evolved together, reinforcing each other in an ever-increasing spiral" \citep[p 504]{jolly:1966}. 

In the late 1980's, the Social Intelligence hypothesis morphed into the Machiavellian Intelligence hypothesis, so called because individual success within complex societies appeared to be "...most effectively promoted by seeming altruistic, honest, and prosocial" \citep{byrne:1988, byrne:1997b}. While research conducted under this hypothesis has found a close relationship between social skills, group complexity, and brain size \citep{byrne:1997a, byrne:1997b}, the excitement garnered in the 80's and 90's over this approach has since dwindled. Key questions focused on what mental representations individuals formed about others' relationships and perspectives, including whether they participated in deliberate deception of conspecifics, and whether they attributed and acted upon surmised belief states of others \citep{byrne:1988, premack:1978, forster:2012}. The focus on using behavior as a "window into" the internal representational states of subjects was echoed throughout first generation cognitive science, but has since been abandoned by many researchers in favor of a more biological account of cognition. 

Over the last thirty years, the field of cognitive science has seen a revolution of methodological approaches to the study of cognition. As more is understood about the continuity of brains, bodies, and worlds dynamically coupled in cognitive processes, researchers have been challenged to examine their units of cognitive analyses, question the absolute centrality of the individual mind in cognitive systems, and increasingly move towards an understanding of cognition as biological processes continuously produced by richly interdependent ecological webs. % Citations

A fundamental insight from this revolution is that social interactions are not simply \textit{residua} of brain-bound cognitive events; but are themselves cognitive processes constituted by richly interconnected ecosystems spanning brains, sensorimotor systems, bodies, and social interlocutors \citep{hutchins:1995b,hutchins:2006}. On this view, social interactions are at least as descriptive of such distributed cognitive processes as the recordings we might take of an individual's neural activity. The careful description of such interactions constitute data points on cognitive events that can be taken on their own right, and may also be used to carefully generate hypotheses about the kinds of sensorimotor and perceptual capabilities a set of organisms must possess of in order to produce the observed social interactions \citep{byrne:2001, johnson:2001, johnson:2001b, king:2003, forster:2006, forster:2012}. 

Re-visiting Hinde's levels with this insight in mind we find renewed methodological traction for the study of the relationship between social and cognitive complexity. Rather than the invisible box left off Hinde's handy diagram, we find that cognitive events are baked right in: they are the interactions. In this dissertation, my aim is to explore and explicate the relationship between social complexity, taken as a quantifiable property that describes a particular kind of patterning of relationships of in an animal society, and cognitive complexity, taken as the context, process, and repertoire of social interaction. 

\section{Organization}

This thesis is organized into two major sections of analysis. In Part A (Chapter 2), I use network analysis to describe the structure, dynamics, and complexity of third-order alliances. In Part B (Chapters 3A and 3B), I use behavioral sampling to describe the context and production of interactions, and examine how such behaviors are used to mediate their social relationships. Chapter 4 investigates the relationship between findings in Part A and Part B, to explicitly address the links between social and cognitive complexity revealed by the undertaken analyses. Major claims for each chapter are summarized below. 

\vspace{3mm}\noindent
\emph{Part A. Network Analysis \& Social Complexity (Ch. 2)}\\
This chapter examines the claim that the third-order alliance network is structurally complex. Evidence supporting this claim includes the following findings:
\begin{easylist}[itemize]
& Individual differences in association patterns indicate the presence of social roles
& Second-order alliances show structural variability and functional similarity
& The third-order network shows quantitative complexity
& Temporal dynamics indicate the ability to adapt to change at all levels of alliance formation. 
& Second-order alliances with more heterogenous relationships (higher entropy) are more resilient to change
\end{easylist}

\vspace{3mm}\noindent
\emph{Part B. Behavioral Analysis \& Cognitive Complexity (Ch. 3A, Ch 3B)}\\
These chapters examine the hypothesis that dolphins use social interactions to mediate alliance relationships. Evidence supporting this claim includes:

\begin{easylist}[itemize]
& During 1st-order consortship contexts, males who have stronger relationships engage in higher rates of social affiliation and coordination (synchrony, formations, petting) than males who have weaker relationships. Males who have weaker relationships tend engage in higher rates of display.
& At 3rd-order group fusions, 3rd-order allies engage in higher rates of SC than five-plus minutes post-fusion.
& When the directionality of events can be distinguished, allies direct petting towards individuals with higher individual consortship rates than themselves.
\end{easylist}

\vspace{3mm}\noindent
\emph{The Link Between Social and Cognitive Complexity (Ch. 4)} \\
This chapter examines the claim that social and cognitive Complexity are explicitly linked, and that link is brought to bear by what allies do when they interact, as well as when, with whom, and how such interactions occur:
\begin{easylist}[itemize]
& When: social contexts
&& Allies engage in different rates of social coordination depending on the social context (comparison of rates of SC across 3 contexts)
&& Embeddedness: allies engage in multiple kinds of interactions simultaneously: e.g. during a third-order fusion event, individuals may simultaneously be engaged in an overall group heading and direction, coordinating spatial and temporally with allies around a female consort, and interacting with third-order allies. These overlapping behavioral states show functional segregation and local integration.
& What: variation in kinds of interactions used
&& The different kinds of interactions (haptic, temporal, visual, and aural) that dolphins produce are semiotic resources.
&& Different relationships are mediated through different sets of semiotic resources (comparison of rates of SC for first-, second-, and third- order allies; also liaisons/facilitators)
&& Liaisons and facilitators (determined by analysis of social roles in I.) show different interaction profiles than non-liaisons / non-facilitators.
& How: the process of achieving / modulating / sustaining coordination
&& The process of coordination is dynamic and emergent: it is co-regulated through micro- changes in the attentional states of the interlocutors.
\end{easylist}

% Why is this observation compelling? 
\section{Notes about the 23 September 2010 Observation}

This event is compelling because it illustrates and otherwise suggests the nature of the social and cognitive complexities that exist in a radically different species than our own.  Attributes of all three levels of alliance formation among the adult male bottlenose dolphin in Shark Bay are discernible in this event: 
\begin{itemize}
\item{The first-order PD alliance PRI WAB NAT were seen together in the absence of a female consort. } %Known
\item{The first-order WC alliances WOW PIK MYR and LAT GRI VEE were both consorting females at the start of the observation, as well demonstrating the fission-fusion dynamic typical of second-order alliances during the peak mating season, where close proximity allows the males to join together to defend their consortships readily, as happened in this event.} %Known, FIZ/FUZ, DEFENSE/OFFSENSE
\item{A seventh WC male, AJA, who was not involved in either consortship, joined his group as soon as it was under attack, with no indication (in this event or in long-term project records of other similar events) of immediate benefit.} % social or reproductive ?. % DELAYED BENEFITS %RCC 
\item{The second-order WC moved both in unison, as sub-groups, and as individuals during the fight against AM-HG, and all seven males were present from start to finish.} % COMPLEXITY, SOCIAL COORDINATION
\item{Midway through the fight, the WC separated, made extremely unusual vocalizations, and re-grouped before re-engaging with AM-HG. Might this represent a kind of tactic among males with a very long history of working together in such encounters?} % SOUND, TACTICS, PRACTICE?, SOCIAL COORDINATION. COMMUNICATION.
\item{The petting we observed among the WC after the fight was likely conciliatory in nature, and may be a mechanism for maintaining alliance relationships in periods of instability. } %INTERACTIONS
\item{AM and HG moved in unison as a third-order alliance to successfully attack the WC, arriving and departing as a coherent group, and with all males engaged in physically intense and potentially dangerous encounter. For the HG there were no immediate benefits, and the AM trio who did not end up with PHA risked the loss of their consortship.} % RISK, THIRD-ORDER, SOCIAL COORDINATION, THE BENEFIT IS THE RELATIONSHIP
\item{This is the only third-order association between the AM and HG alliances on record, including the time prior to this event and in five years of data collection since. AM are typically only sighted during the peak-mating season, when they appear to move in from an un-surveyed area north of the research site, and HG were still a maturing group at the time of this observation. Either could explain the apparent temporary nature of their third-order alliance, but the possibility that at least some third-order alliances are temporary is left open.} % TEMPORAL DYNAMICS. FUNCTION AND STABILITY OF THIRD ORDER ALLIANCES (COALITIONS?), DELAYED BENEFITS
\item{In all, five second-order alliances were present for this event. Two second-order alliances (Friends of AM, KS) were observed near the fight, perhaps ready to join, or even playing a role that went un-detected by boat-based human observers. Long-term project records suggest these groups may have been third-order associates of AM and WC, respectively. Additionally, in an open and dynamic social network such as this one, there are likely benefits to keeping track of relationships among second-order alliances, similar to the manner in which other species keep track of a changing dominance hierarchy within their group.} % Check, extend, site. this is cool. % DOMINANCE HIERARCHY, FIZ/FUZ 
\item{Structural and functional indications of social complexity are explicitly present in this observation: WC first-order alliances were participating in consortships, fizzing and fusing in these sub-groups prior to and during the fight. The original six WC were joined by a seventh WC during a reproductive competition in which all WC males fought against the AM-HG alliance. The second-order AM-HG groups formed a temporary coalition, moving and fighting in unison against the WC. During the fight itself the consorting WC males played 3-4 roles each: a consort proximal to and defending his mate, a first-order ally coordinating with the other two males in his consortship trio, a second-order ally coordinating activity among the other six males in the WC group, and perhaps even a third-order ally when engaged with the KS. Eventually, the AM males would play all these roles as well: a third-order ally coordinating his behavior with the other 11 AM-HG males, a second-order ally when (possibly precipitating) the AM and HG split, a first-order ally when coordinating with two other males to establish a new consortship with Phantom, and a consort to Phantom. These examples naturally separate into temporal and functional units, but during the fight we observed individuals moving fluidly between roles, existing in multiple roles in a single instance.} % CMJ complexity. ROLES, COMPLEXITY
\end{itemize}

% HEIRARCHIES ARE NOT WITHIN ALLIANCES, THEY ARE ***AMONG*** ALLIANCES. THAT'S WHY THEY HAVE TO KEEP TRACK. CONSIDER LOCAL INFORMATION EXCHANGE AMONG ALLIES DIFFERENTLY. INFO TRANSFER WITHIN GROUPS = SOMEONE SHOULD WATCH. SHOULD OTHERS CONFIRM?
% A kind of enconomics. risk, danger, energy --> future benefits? Don't get into this too much. 

% This is the kind of event that can generate tens, if not hundreds of questions. In this dissertation, I will try to tackle the few that stand out to me as particularly relevant. %Why? Relevant in what context? 

% Social complexity
% Temporal dynamics
% Processes of mediation
% Social coordination and reproductive success


%%%%%%%%%%%%%%%%%%%%%
% REFERENCES % 
%%%%%%%%%%%%%%%%%%%%%
\clearpage
\bibliographystyle{apalike}
\bibliography{/Users/Whitney/Dropbox/Articles/bibliography-wf}

%%%%%%%%%%%%%%%%%%%%%
\end{document}